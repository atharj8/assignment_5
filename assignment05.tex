\documentclass[tikz,border=2pt,png]{article}
\usepackage{tikz}
\usepackage{amsmath}
\usepackage{amssymb}
\usepackage{pgfplots}
\begin{document}
\title{ASSIGNMENT 5}
\author{Athar Javed}
\date{\today}
\maketitle

\begin{itemize}
\item{\textbf{Exercise 2.58:}}\\

Draw a pair of tangents to a circle of radius 5 cm which are inclined to each other at an angle of 60^{\circ}:\\


\item{\textbf{Solution:}}\\

$ 

Given, Radius of circle=r=5cm\\
$

Now,\\

\textit{Steps of Construction are:}\\
$

1: Draw circle with centre O and radius OA=5 cm.\\

2: Mark another point B on the circle such that {\angle}AOB=120^{\circ},\\

supplementary to the angle between the tangents. Since the angle between the \\

tangents to be constructed is 60^{\circ}.\\

\therefore{} & \hspace{3 cm}{\angle}AOB=180^{\circ}-60^{\circ}=120^{\circ}.\\

3: Construct angles of 90^{\circ},$ $ 

at A and B and extend the lines so as to intersect at point P.\\

4: Thus, AP and BP are the required tangents to the circle.\\
$

Now, the figure of constructed tangents is given below,\\
 

 \begin{tikzpicture}[scale=0.65]
 \draw(-10,0)--(10,0) node[anchor=north west]{x axis};
 \draw(0,-7)--(0,7) node[anchor=south east]{y axis};
 \draw[blue] (0,0)--(4.3,-2.7);
 \draw[blue] (0,0)--(0,5);
 \node at (4.3,-3){$A$};
 \node at (-0.3,5.3){$B$}; 
 \node at (0.3,0.3){$O$};
 \draw (0,0) circle (5 cm);
 \node at (9,5.3){P};
 \draw[red] (4.27,-2.7)--(9,5);
 \draw[red] (0,5)--(9,5);
 
 \end{tikzpicture}\\


BP and AP are the pair of tangents to a given circle.\\

\newpage
\item{\textbf{Question 2.59:}}\\

Draw a line segment AB of length 8 units. Taking A as centre, draw a circle of radius 4 units and taking B as centre, draw another circle of radius 3 units. Construct tangents to each circle from the centre of the other circle:

\item{\textbf{Solution:}}\\

Given the radii of two concentric circles of radius 4 units and 6 units respectively,\\

\textit{Steps of Construction:}\\
$

1: Draw a line segment AB= 8units.\\

2: Draw a circle of radius 4 units, taking A as its centre.\\

3: Draw a circle of radius 3 units, taking B as its centre.\\

4: Now, Draw the tangents to each circle from the centre of other circle.\\ 
$\\
 
   From the construction required tangents are given below:\\
   

 \begin{tikzpicture}[scale=0.65]
 \draw(-12,0)--(12,0) node[anchor=north west]{x axis};
 \draw(0,-5)--(0,5) node[anchor=south east]{y axis};
 \draw (0,0)--(8,0);
 \node at (-0.3,0.3){$A$};
 \node at (8.3,0.3){$B$};
 \draw (0,0) circle (4 cm);
 \draw (8,0) circle (3 cm);
 \draw[red] (8,0)--(2.3,3.3);
 \draw[red] (8,0)--(2.3,-3.3);
 \node at (2.4,3.6){$C$};
 \node at (2.4,-3.6){$F$};
 \draw[red] (0,0)--(7.2,2.9);
 \draw[red] (0,0)--(7.2,-2.9);
 \node at (7.3,3.2){$D$};
 \node at (7.3,-3.2){$E$};
 \end{tikzpicture}\\

 
 AD, AE and BC,BF are the two pairs of tangents to each circle from the centre of other circle.
\end{itemize}
\end{document}